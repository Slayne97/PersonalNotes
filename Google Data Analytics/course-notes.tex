\documentclass[a4paper]{article}
\usepackage[utf8]{inputenc}

%_______________________________________________________________
%                    Packages needed:
%_______________________________________________________________
\usepackage{tikz, adjustbox}
\usepackage[most]{tcolorbox}
\usepackage{xcolor}
\usepackage{wrapfig}
\newcommand*{\plogo}{\fbox{$\mathcal{PL}$}} % Generic dummy publisher logo
\usepackage[utf8]{inputenc} % Required for inputting international characters
\usepackage[T1]{fontenc} % Output font encoding for international characters
\usepackage{stix} % Use the STIX fonts
\usepackage[utf8]{inputenc}
\usepackage{xcolor}
\usepackage[explicit]{titlesec}
\usepackage{soul}
\usepackage[a4paper, margin=1in]{geometry}

%................................................................
%
%           Defining colors for sticky notes:
%_______________________________________________________________
% Yellow:
\definecolor{BgYellow}{HTML}{FFF59C}
\definecolor{FrameYellow}{HTML}{F7A600}
% Pink:
\definecolor{BgPink}{HTML}{EF6FA7}
\definecolor{FramePink}{HTML}{E5446E}
% Green:
\definecolor{BgGreen}{HTML}{C7D92D}
\definecolor{FrameGreen}{HTML}{89B23B}
% Blue:
\definecolor{BgBlue}{HTML}{45BEE9}
\definecolor{FrameBlue}{HTML}{31A8C9}
% White:
\definecolor{BgWhite}{HTML}{D8D8D8}
\definecolor{FrameWhite}{HTML}{7F7F7F}
% Brown:
\definecolor{BgBrown}{HTML}{8E7A45}
\definecolor{FrameBrown}{HTML}{6B5B32}
%................................................................
%
%                   Dummy text package:
%_______________________________________________________________
\usepackage{lipsum}
%................................................................
%
%                   NB command:
%_______________________________________________________________
\usepackage{contour}
\newcommand{\NB}{\contour{black}{\textbf{{\large\sffamily\color{red}NB}}}\textbf{\large\sffamily: }}
%................................................................
%
%               Defining Sticky note boxes:
%_______________________________________________________________
% Yellow Sticky Note (YStkyNote):
\newtcolorbox{YStkyNote}[1][]{%
    enhanced,
    before skip=2mm,after skip=2mm, 
    width=0.4\textwidth, % width of the sticky note
    boxrule=0.2mm,
    colback=BgYellow, colframe=FrameYellow, % Colors
    attach boxed title to top left={xshift=0cm,yshift*=0mm-\tcboxedtitleheight},
    varwidth boxed title*=-3cm,
    % The titlebox:
    boxed title style={frame code={%
        \path[left color=FrameYellow,right color=FrameYellow,
        middle color=FrameYellow]
        ([xshift=-0mm]frame.north west) -- ([xshift=0mm]frame.north east)
        [rounded corners=0mm]-- ([xshift=0mm,yshift=0mm]frame.north east)
        -- (frame.south east) -- (frame.south west)
        -- ([xshift=0mm,yshift=0mm]frame.north west)
        [sharp corners]-- cycle;
        },interior engine=empty,
    },
    sharp corners,rounded corners=southeast,arc is angular,arc=3mm,
    % The "folded paper" in the bottom right corner:
    underlay={%
        \path[fill=BgYellow!80!black] ([yshift=3mm]interior.south east)--++(-0.4,-0.1)--++(0.1,-0.2);
        \path[draw=FrameYellow,shorten <=-0.05mm,shorten >=-0.05mm,color=FrameYellow] ([yshift=3mm]interior.south east)--++(-0.4,-0.1)--++(0.1,-0.2);
        },
    drop fuzzy shadow, % Shadow
    fonttitle=\bfseries, 
    title={#1}
}
% Pink Sticky Note (PStkyNote):
\newtcolorbox{PStkyNote}[1][]{%
    enhanced,
    before skip=2mm,after skip=2mm, 
    width=0.4\textwidth, % width of the sticky note
    boxrule=0.2mm, 
    colback=BgPink, colframe=FramePink, % Colors
    attach boxed title to top left={xshift=0cm,yshift*=0mm-\tcboxedtitleheight},
    varwidth boxed title*=-3cm,
    % The titlebox:
    boxed title style={frame code={%
        \path[left color=FramePink,right color=FramePink,
        middle color=FramePink]
        ([xshift=-0mm]frame.north west) -- ([xshift=0mm]frame.north east)
        [rounded corners=0mm]-- ([xshift=0mm,yshift=0mm]frame.north east)
        -- (frame.south east) -- (frame.south west)
        -- ([xshift=0mm,yshift=0mm]frame.north west)
        [sharp corners]-- cycle;
        },interior engine=empty,
    },
    sharp corners,rounded corners=southeast,arc is angular,arc=3mm,
    % The "folded paper" in the bottom right corner:
    underlay={%
        \path[fill=BgPink!80!black] ([yshift=3mm]interior.south east)--++(-0.4,-0.1)--++(0.1,-0.2);
        \path[draw=FramePink,shorten <=-0.05mm,shorten >=-0.05mm,color=FramePink] ([yshift=3mm]interior.south east)--++(-0.4,-0.1)--++(0.1,-0.2);
        },
    drop fuzzy shadow, % Shadow
    fonttitle=\bfseries, 
    title={#1}
}
% Green Sticky Note (GStkyNote):
\newtcolorbox{GStkyNote}[1][]{%
    enhanced,
    before skip=2mm,after skip=2mm, 
    width=0.4\textwidth, % width of the sticky note
    boxrule=0.2mm,
    colback=BgGreen, colframe=FrameGreen, % Colors
    attach boxed title to top left={xshift=0cm,yshift*=0mm-\tcboxedtitleheight},
    varwidth boxed title*=-3cm,
    % The titlebox:
    boxed title style={frame code={%
        \path[left color=FrameGreen,right color=FrameGreen,
        middle color=FrameGreen]
        ([xshift=-0mm]frame.north west) -- ([xshift=0mm]frame.north east)
        [rounded corners=0mm]-- ([xshift=0mm,yshift=0mm]frame.north east)
        -- (frame.south east) -- (frame.south west)
        -- ([xshift=0mm,yshift=0mm]frame.north west)
        [sharp corners]-- cycle;
        },interior engine=empty,
    },
    sharp corners,rounded corners=southeast,arc is angular,arc=3mm,
    % The "folded paper" in the bottom right corner:
    underlay={%
        \path[fill=BgGreen!80!black] ([yshift=3mm]interior.south east)--++(-0.4,-0.1)--++(0.1,-0.2);
        \path[draw=FrameGreen,shorten <=-0.05mm,shorten >=-0.05mm,color=FrameGreen] ([yshift=3mm]interior.south east)--++(-0.4,-0.1)--++(0.1,-0.2);
        },
    drop fuzzy shadow, % Shadow
    fonttitle=\bfseries, 
    title={#1}
}
% Blue Sticky Note (BStkyNote):
\newtcolorbox{BStkyNote}[1][]{%
    enhanced,
    before skip=2mm,after skip=2mm, 
    width=0.4\textwidth, % width of the sticky note
    boxrule=0.2mm,
    colback=BgBlue, colframe=FrameBlue, % Colors
    attach boxed title to top left={xshift=0cm,yshift*=0mm-\tcboxedtitleheight},
    varwidth boxed title*=-3cm,
    % The titlebox:
    boxed title style={frame code={%
        \path[left color=FrameBlue,right color=FrameBlue,
        middle color=FrameBlue]
        ([xshift=-0mm]frame.north west) -- ([xshift=0mm]frame.north east)
        [rounded corners=0mm]-- ([xshift=0mm,yshift=0mm]frame.north east)
        -- (frame.south east) -- (frame.south west)
        -- ([xshift=0mm,yshift=0mm]frame.north west)
        [sharp corners]-- cycle;
        },interior engine=empty,
    },
    sharp corners,rounded corners=southeast,arc is angular,arc=3mm,
    % The "folded paper" in the bottom right corner:
    underlay={%
        \path[fill=BgBlue!80!black] ([yshift=3mm]interior.south east)--++(-0.4,-0.1)--++(0.1,-0.2);
        \path[draw=FrameBlue,shorten <=-0.05mm,shorten >=-0.05mm,color=FrameBlue] ([yshift=3mm]interior.south east)--++(-0.4,-0.1)--++(0.1,-0.2);
        },
    drop fuzzy shadow, % Shadow
    fonttitle=\bfseries, 
    title={#1}
}
% White Sticky Note (WStkyNote):
\newtcolorbox{WStkyNote}[1][]{%
    enhanced,
    before skip=2mm,after skip=2mm, 
    width=0.4\textwidth, % width of the sticky note
    boxrule=0.2mm,
    colback=BgWhite, colframe=FrameWhite, % Colors
    attach boxed title to top left={xshift=0cm,yshift*=0mm-\tcboxedtitleheight},
    varwidth boxed title*=-3cm,
    % The titlebox:
    boxed title style={frame code={%
        \path[left color=FrameWhite,right color=FrameWhite,
        middle color=FrameWhite]
        ([xshift=-0mm]frame.north west) -- ([xshift=0mm]frame.north east)
        [rounded corners=0mm]-- ([xshift=0mm,yshift=0mm]frame.north east)
        -- (frame.south east) -- (frame.south west)
        -- ([xshift=0mm,yshift=0mm]frame.north west)
        [sharp corners]-- cycle;
        },interior engine=empty,
    },
    sharp corners,rounded corners=southeast,arc is angular,arc=3mm,
    % The "folded paper" in the bottom right corner:
    underlay={%
        \path[fill=BgWhite!80!black] ([yshift=3mm]interior.south east)--++(-0.4,-0.1)--++(0.1,-0.2);
        \path[draw=FrameWhite,shorten <=-0.05mm,shorten >=-0.05mm,color=FrameWhite] ([yshift=3mm]interior.south east)--++(-0.4,-0.1)--++(0.1,-0.2);
        },
    drop fuzzy shadow, % Shadow
    fonttitle=\bfseries, 
    title={#1}
}
% Brown Sticky Note (BrStkyNote):
\newtcolorbox{BrStkyNote}[1][]{%
    enhanced,
    before skip=2mm,after skip=2mm, 
    width=0.4\textwidth, % width of the sticky note
    boxrule=0.2mm,
    colback=BgBrown, colframe=FrameBrown, % Colors
    attach boxed title to top left={xshift=0cm,yshift*=0mm-\tcboxedtitleheight},
    varwidth boxed title*=-3cm,
    % The titlebox:
    boxed title style={frame code={%
        \path[left color=FrameBrown,right color=FrameBrown,
        middle color=FrameBrown]
        ([xshift=-0mm]frame.north west) -- ([xshift=0mm]frame.north east)
        [rounded corners=0mm]-- ([xshift=0mm,yshift=0mm]frame.north east)
        -- (frame.south east) -- (frame.south west)
        -- ([xshift=0mm,yshift=0mm]frame.north west)
        [sharp corners]-- cycle;
        },interior engine=empty,
    },
    sharp corners,rounded corners=southeast,arc is angular,arc=3mm,
    % The "folded paper" in the bottom right corner:
    underlay={%
        \path[fill=BgBrown!80!black] ([yshift=3mm]interior.south east)--++(-0.4,-0.1)--++(0.1,-0.2);
        \path[draw=FrameBrown,shorten <=-0.05mm,shorten >=-0.05mm,color=FrameBrown] ([yshift=3mm]interior.south east)--++(-0.4,-0.1)--++(0.1,-0.2);
        },
    drop fuzzy shadow, % Shadow
    fonttitle=\bfseries, 
    title={#1}
}

% 
\definecolor{titleblue}{HTML}{2A202C}

\newbox\TitleUnderlineTestBox
\newcommand*\TitleUnderline[1]
  {%
    \bgroup
    \setbox\TitleUnderlineTestBox\hbox{\colorbox{titleblue}\strut}%
    \setul{\dimexpr\dp\TitleUnderlineTestBox-.3ex\relax}{.3ex}%
    \ul{#1}%
    \egroup
  }
\newcommand*\SectionNumberBox[1]
  {%
    \colorbox{titleblue}
      {%
        \makebox[2.5em][c]
          {%
            \color{white}%
            \strut
            \csname the#1\endcsname
          }%
      }%
    \TitleUnderline{\ \ \ }%
  }
\titleformat{\section}
  {\Large\bfseries\sffamily\color{titleblue}}
  {\SectionNumberBox{section}}
  {0pt}
  {\TitleUnderline{#1}}
\titleformat{\subsection}
  {\large\bfseries\sffamily\color{titleblue}}
  {\SectionNumberBox{subsection}}
  {0pt}
  {\TitleUnderline{#1}}




\begin{document}
    \begin{titlepage}
        \raggedleft
        \rule{1pt}{\textheight}
        \hspace{0.05\textwidth}
        \parbox[b]{0.75\textwidth}{
            {\Huge\bfseries \textcolor[HTML]{2A202C}{Google Data Analytics}\\[0.5\baselineskip] \textcolor[HTML]{2A202C}{Professional Certificate}}\\[2\baselineskip]
            {\large\textit{\textcolor[HTML]{2A202C}{Course Notes}}}\\[4\baselineskip]
            {\Large\textsc{\textcolor[HTML]{2A202C}{Jesus Cardenaz}}}
            
            \vspace{0.5\textheight}
        }
    \end{titlepage}

    \section{Concepts}

    Thhe course will focus on effective questioning, problem-solving, types of data, 
    structured thinking, and communication strategies. 

    \begin{itemize}
        \item Effective and Ineffective Questions: 
            \begin{itemize}
                \item Effective questions lead to insights and help solve business problems
                \item Ineffective questions hinder the data analysis process.
            \end{itemize}
        \item Data Analysis Phases 
            \begin{enumerate}
                \item Ask: Defining the problem, understanding stakeholder expectations.
                \item Prepare: Gathering and organizing data for analysis. 
                \item Process: Cleaning, transforming, and formatting data. 
                \item Analyse: Applying statistical and analytical techniques. 
                \item Share: Presenting findings and insighits to stakeholders.
                \item Act: Taking action based on the analysis.
            \end{enumerate}
        \item Structured Thinking
            \begin{itemize}
                \item Structured thinking involves recognizing the current problem, organizing information, identifying gaps and opportunities, and determining options.
                \item It breaks down complex problems into smaller steps to find logical solutions.
            \end{itemize}
        \item Data Collection Methods
            \begin{itemize}
                \item First-party data: Data collected by an individual or group using their own resources.
                \item Second-party data: Data collected by a group directly from its audience and then sold.
                \item Third-party data: Data collected from outside sources that were not collected directly.
            \end{itemize}

        \item Data Collection Considerations
            \begin{itemize}
                \item How the data will be collected
                \item Choose data sources
                \item Decide what data to use
                \item Select the right data type
                \item Determine the time frame 
            \end{itemize}

        \item Types of Data
            \begin{itemize}
                \item Qualitative Data: Descriptive data that cannot be easily expressed using number, such as movie titles and cast members.
                \item Quantitative Data: Measurable or countable data expressed as numbers, such as budget and box office revenue.
                \item Discrete Data: Counted data with a limited number of values.
                \item Continuous Data: Measured data that can be expressed as decimals with multiple decimal places.
                \item Nominal Data: Qualitative data categorized without a set order, such as responses to a yes/no question.
                \item Ordinal Data: Qualitative data with a set order or scale, such as ranking a movie from 1 to 5.
                \item Internal Data: Data that resides within a company's own systems and is collected using their own methods.
                \item External Data: Data generated outside of an organization, often collected from various sources.
                \item Structured Data: Organized in a specific format, such as spreadsheets or databases, making it easily searchable and analysis-ready.
                \item Unstructured Data: Data that lacks a clear organization, such as audio and video files, and does not fit into rows and columns.
            \end{itemize}
        \item Data Modeling 
            \begin{itemize}
                \item Data Models are visual representations that organizes data elements and their relationships, ensuring consistency and providing a framework for understanding the data.
            \end{itemize}

        \item Types of data modeling 
            \begin{itemize}
                \item Conceptual: Gives a high-level view of the data structure, such as how data interacts across an organization. It does no tcontain technical details.
                \item Logical: Focuses on the technical details of a database such as relationships, attributes and entities. 
                \item Physical: Depicts how a database operates. A physical data model defines all entities and attributes used.
            \end{itemize}

        \item Bias
            \begin{itemize}
                \item Sampling Bias: When a sample ins't representative of the population as a whole.
                \item Unbiased Sampling: When a sample is representative of the population being measured.
                \item Observer Bias: The tendency fot different people to observe thing differently
                \item Interpretation Bias: The tendendcy to always interpret ambiguous situations in a positive or negative way.
                \item Confirmation Bias: Is the tendency to search for or interpret information in a way that confirms pre-existing beiefs.
            \end{itemize}

        \item Identifying good data sources, a good data source needs to be:
            \begin{itemize}
                \item Reliable
                \item Original
                \item Comprehensive
                \item Current 
                \item Cited
            \end{itemize}


        \item Data Ethics 
            \begin{itemize}
                \item Data Ethics: Well-founded standards of right and wrong that dictate how data is collected, shared, and used.
                \item Ownership: The idea that individuals own the raw data they provide and have control over its usage, processing, adn sharing.
                \item Transaction transparency: All data processing activities and algorithms should be explainable and understood by the individuals who provice their data to ensure fairness and address concerns of data bias.
                \item Consent: The right of individuals to know explicit details about how and why their data will be used before agreeing to provide it.
                \item Currency: Awareness of financial transactions resulting from the use of presonal data and the ability to opt out.
                \item Privacy: Protecting and securing personal information and data. 
                \item Openness: Maing the data process transparent and accesisble to individuals.
            \end{itemize}

        \item Data Subject 
            \begin{itemize}
                \item An individual whose personal data is being collected, processed, or stored.
            \end{itemize}

        \item  Data often anonymized
            \begin{itemize}
                \item Thelephone numbers
                \item Names
                \item License plates
                \item Social Security numbers
                \item IP addresses
                \item Medical records
                \item email addresses
                \item photographs
                \item Account numbers
            \end{itemize}


        \item Databases
            \begin{itemize}
                \item Database Normalization: Is the process of organizing data in a relational database, reating tables, and establishing relationships between those tables.
                \item 
            \end{itemize}

        \item  Metadata
            \begin{itemize}
                \item Descriptive metadata: Describes a piece of data and can be used to identify it at a later point in time. (ID, Names)
                \item Structural metadata: Indicates how a piece of data is organized and whether it is part of one, or more than one, data collection (Categories, Chapters, Relationships)
                \item Administrative metadata: Indicates the technical source of a digital asset. (File Type, Date, etc)
            \end{itemize}

        \item Metadata Repositories - A database specifically created to store metadata
            \begin{itemize}
                \item Describes the state and location of themetadata
                \item Describes the structures of the tables inside
                \item Describes how the data flows through the repository
                \item Keep track of who accesses the metadata and when
            \end{itemize}

        \item Data ingegrity
            \begin{itemize}
                \item The accuracy, completeness, consistency, and trustworthiness of data throughout its lifecylce.
            \end{itemize}

        \item Verification 
            \begin{itemize}
                \item A process to confirm that a data-cleaning effort was well-executed and the resulting data is accurate and reliable.
            \end{itemize}
            
        \item Big Picture when verifying data-cleaning 
            \begin{enumerate}
                \item Consider the business problem.
                \item Consider the goal of the proyect.
                \item Consider the data.
            \end{enumerate}

        \item The 4 Phases of Analysis
            \begin{itemize}
                \item Organize data: Involves arrangin dat in a structured manner to facilitate further exploration and understanding.
                \item Format and adjust: Data is modified and prepared for analysis by applying filters, sorting, or other adjustments to make it more easily digestible and accessible. 
                \item Get Input from others: Seeking feedback, opinions, and perspectives from other individuals to gain insights and considerations that may help in the analysis process.
                \item Transform Data: Relationships and patterns within the data are identified, and calculations are made based on the available data to derive meaningful insights and make informed decisions. 
            \end{itemize}
            
        \item Elements of a successful visualization:
            \begin{itemize}
                \item Information/Data: The underlying data or information being visualized.
                \item Story: The narrative or concept that adds meaning and interest to the data.
                \item Goal: The purpose or function of the visualization, making the data useful and usable.
                \item Visual form: The visual design and structure of the visualization.
            \end{itemize}
        
        \item Elements for effective visuals 
            \begin{itemize}
                \item Clear meaning: Clearly comunicating the intended insight or message to the audience.
                \item Sophisticated use of contrast: Utilizing visual context and contrast to highlight the most important data in a visualization
                \item Refined execution: Attention to detail and effective use of visual elements, such lines, shapes, colors, value, spce and movement, in a visualizations
            \end{itemize}



        \item Data Storytelling steps 
            \\\\ Data Storytelling is communicating the meaning of a dataset with vsuals and narrative that is
            customized for a particular audience.
            
            \begin{itemize}
                \item Engage your audience
                \item Create compelling visuals
                \item Tell the story in an interesting narrative.
            \end{itemize}

        \item The McCandless Method
            \begin{enumerate}
                \item Introduce the graphic by Names
                \item Answer obvious questions before they're asked
                \item State the insight of your graphic
                \item Call out data to support that insight
                \item Tell your audience why it matters
            \end{enumerate}

        \item What not to do in a presentation 
            \begin{itemize}
                \item No story, no logical flows
                \item No titles 
                \item Too much text 
                \item Hard to understand
                \item Uneven and inconsistent format, no theme 
                \item No conclusion or recommendation
            \end{itemize}

        \item Responding to possible objections
            \begin{itemize}
            \item Communicate any assumptions
            \item Explain why your analysis might be different than expected 
            \item Acknowledge that those objections are valid and take steps to investigate further
        \end{itemize}

        \item Types of Objections
            \begin{itemize}
                \item About the data 
                    \begin{itemize}
                        \item Whre you got the data? 
                        \item What systems it came from? 
                        \item What transformations happened to it?
                    \end{itemize}
                
                    \item About your analysis 
                    \begin{itemize}
                        \item Is your analysis reproducible?
                        \item Who did you get feedback from?
                    \end{itemize}

                \item About your findings
                    \begin{itemize}
                        \item Do these findings exist in previous time periods?
                        \item Did you control for the differences in your data?
                    \end{itemize}

            \end{itemize}




    \end{itemize}





    


\end{document}

    % Stick Notes examples 
    % % Put the sticky note in a wrapfigure to have text wrap around it.
    % \begin{wrapfigure}{L}{0.45\textwidth}
    %     \begin{YStkyNote}[Note 1]
    %         This text is \emph{important}. Here is an useful equation:
    %         \begin{align}
    %             \sin (x) \approx x
    %         \end{align}
    %     \end{YStkyNote}
    % \end{wrapfigure}

    % \begin{wrapfigure}{R}{0.45\textwidth}
    %     \begin{PStkyNote}[Note 2]
    %     Here is some more text. This is useful information you need to know:
    %     \begin{itemize}
    %         \item sin approximation valid \textbf{only} for small angles
    %         \item 2 radians is \textit{not} a small angle
    %     \end{itemize}
    %     \end{PStkyNote}
    % \end{wrapfigure}

    % \begin{wrapfigure}{L}{0.45\textwidth}
    %     \begin{GStkyNote}[Note 3]
    %     \NB do not forget this!
    %     \end{GStkyNote}
    % \end{wrapfigure}

    % \begin{wrapfigure}{R}{0.45\textwidth}
    %     \begin{BStkyNote}[Note 4]
    %     This will be on the final exam!
    %     You better \emph{study} hard!
    %     \end{BStkyNote}
    % \end{wrapfigure}

    % \begin{wrapfigure}{L}{0.45\textwidth}
    %     \begin{WStkyNote}[Note 5]
    %         \begin{equation}
    %             pV=Nk_BT
    %         \end{equation}
    %     \end{WStkyNote}
    % \end{wrapfigure}

    % \begin{wrapfigure}{R}{0.45\textwidth}
    %     \begin{BrStkyNote}[Note 6]
    %     Type \verb+\NB+ to get the \NB text.
    %     \end{BrStkyNote}
    % \end{wrapfigure}